\chapter{Introdução}

%1) Começar escrevendo sobre a importância de sistemas embarcados;

Sistemas embarcados estão instalados e controlam os mais diversos equipamentos hoje em dia, eles estão mais presentes do que nos damos conta. Conforme eles estão sendo cada vez mais empregados, sua demanda aumenta também, bem como seu poder de processamento. Máquinas que antes utilizavam vários pequenos microcontroladores trabalhando independentemente, hoje podem ser gerenciados com melhor custo-benefício por um sistema de maior desempenho.%, como há exemplos em carros e aviões.

%2) Citar que alguns desses sistemas demandam maior processamento que só pode ser satisfeito por um hardware dedicado ou por um processador multicore;

Hoje sistemas embarcados estão empregados em tarefas que exigem a manipulação de grande volume de dados (muitas vezes em tempo real), como na área de comunicação e transmissão de dados. O nome ``sistema embarcado'' já foi sinônimo de pequenos microcontroladores dedicados, mas hoje também há grande demanda por sistemas embarcados de alto desempenho. Uma meio de obter-se processamento de alto desempenho à baixo custo é com a implantação de processadores \emph{multicores}.

Aplicações de tempo real necessitam tem um tempo de resposta previsível, e suas tarefas devem ser executadas sem atrasos (no caso de \emph{hard real time}). Um exemplo são os sistemas de airbag automotivo, que necessitam de uma resposta a um estímulo em um determinado espaço de tempo \cite{nat}. Muitas destas aplicações são feitas diretamente em hardware para se adequar aos requesitos da aplicação (tempo real, gasto energético, etc). Entretanto aplicações feitas diretamente em hardware apresentam problemas em relação à manutenibilidade e possibilidade de atualização \cite{gio}. Estes componentes de hardware poderiam ser centralizados em um sistema embarcado de maior processamento, e alocar tarefas de tempo real para cada uma das funções necessárias ao sistema. Deste modo, a manutenção e custo total do sistema pode ser melhorado, isto enquanto mantem-se todas as restrições temporais.

Outro exemplo são os quadcópteros. Controlar um quadcóptero não é uma tarefa trivial, há uma miríade de fatores a serem considerados. Um quadcóptero precisa balancear quanto empuxo deve ser aplicado à cada hélice para manter-se no ar e fazer manobras, é necessário também que o sistema seja tolerante à faltas para ser seguro. Em caso de um quadcóptero autônomo, os desafios são ainda maiores. Passa a ser necessário geolocalização, detectar à que distancia se está do chão, controle de pouso, etc \cite{leeseshia}.
Não é difícil perceber a grande quantidade de dados a serem processados, e que uma falha em processar estes dados no tempo estimado pode gerar numa falha total do sistema (a queda do quadcóptero, neste caso). Neste exemplo, nota-se como é necessário um sistema operacional previsível, e um hardware capaz de processar rapidamente e paralelamente diversos dados.

%%%%%%%%%%%%%% EPOS %%%%%%%%%%%%%%%%%

%4) Falar da importância de sistemas operacionais de tempo real para sistemas embarcados multicore, principalmente devido ao escalonador;

É natural que estas plataformas de maior desempenho, que consequentemente possuem mais recursos e componentes à serem gerenciados, necessitem de um sistema operacional que se encarregue desta função, e que possam administrar eficientemente estes recursos.
De acordo com um estudo \cite{UBM}, 68\% dos projetos em computação embarcada usam sistemas operacionais, sendo a principal razão para isto é ter garantias de tempo real.

De acordo com este mesmo estudo \cite{UBM}, metade dos projetos de sistemas embarcados usaram mais que um microcontrolador/processador em seu projeto, e 29\% dos processadores usados nestes projetos possuem um \emph{clock} superior à 500 MHz, sendo que a média de \emph{clock} destes projetos é de 485 MHz.
Além disto, tem sido comum o uso de FPGAs\footnote{FPGAs podem ser pensadas como um hardware que pode ser programado pelo usuário.} integradas à estes sistemas, onde 31\% dos projetos usaram FPGAs em 2013.
Dentre estas placas, podemos destacar a Zedboard, que é uma plataforma de bom desempenho com um processador \emph{dualcore} de 666 MHz e FPGA.

%5) Citar o EPOS como primeiro RTOS de código aberto a suportar  os escalonadores global, particionado e agrupado;

Uma peça central de um sistema operacional de tempo real é seu escalonador de processos/\emph{threads}. É neste componente que políticas sobre priorização de tarefas são feitas de modo a garantir que cada tarefa seja cumprida dentro de seu prazo.

EPOS, que é um sistema operacional embarcado de tempo real (RTOS\footnote{Real-Time Operating System}) \emph{multithread}, é o o primeiro RTOS de código aberto a suportar os escalonadores de tempo real global, particionado e agrupado \cite{gio}.

%6) Falar que atualmente o EPOS só oferece suporte ao IA32 (veja comentário abaixo).

Estes fatos demonstram a necessidade de um sistema operacional de tempo real como o EPOS de ter suporte para uma plataforma embarcada multicore. A única arquitetura \emph{multicore} suportada pelo EPOS era a IA32, que não é uma arquitetura que oferece previsibilidade, logo para todos os sistemas embarcados com restrições temporais uma arquitetura multicore com maior previsibilidade torna-se necessário.

O EPOS não possuia suporte para um processador embarcado \emph{multicore}. 
Este trabalho trata de descrever como foi feito o porte
\footnote{Porte é um estrangeirismo da palavra \emph{port}, que, no âmbito da computação, significa o ato de fazer um mesmo programa/sistema/SO funcionar em diferentes ambientes. Por exemplo, fazer um software que antes só funcionava no Linux passar a funcionar em um outro SO (sistema operacional) pode ser considerado um porte. Uma palavra alternativa que poderia ser usada é ``suporte'', entretanto acredito que esta palavra não expresse apropriadamente o que foi feito, já que essa palavra normalmente é associada com um serviço pago de assistência técnica, e o próprio fraseamento do que foi feito se tornaria de mais difícil compreensão e prolixo com esta palavra.}
 deste sistema operacional para a Zedboard, uma plataforma embarcada \emph{multicore}.

\section{Objetivos Gerais}

O objetivo deste trabalho é portar o sistema operacional EPOS e documentar este processo no presente documento. O porte consiste em adaptar os componentes que, dentro da arquitetura do EPOS, são chamados de mediadores de hardware. Cada mediador precisa ser refeito para cada plataforma.
Este trabalho terá certo direcionamento para as pessoas que no futuro precisem fazer um novo porte do EPOS, de modo que elas não precisem passar pelos mesmos problemas e possam, portanto, fazer isto mais eficientemente.

\section{Objetivos Específicos}
Dos objetivos específicos, pode-se organizá-los da seguinte forma:

\begin{enumerate}
    \item Estudo da arquitetura do EPOS, da arquitetura do ARM Cortex A9 e do Zedboard.
    \item Portes
    \begin{enumerate}
        \item Timers
        \item Controlador de Interrupção
        \item MMU
		\item UART
		\item CPU
		\item Inicialização \emph{multicore}
    \end{enumerate}
%    \item Validações
%    \begin{enumerate}
%        \item Escalonadores em single e multicore
%        \item Threads
%        \item Thread Periódica
%        \item Alarmes
%        \item Cronômetro
%        \item Sincronização
%    \end{enumerate}
    \item Documentação
\end{enumerate}
