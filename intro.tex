\chapter{Introdução}

O presente trabalho visa descrever como foi feito o porte\footnote{Porte é um estrangeirismo da palavra \emph{port}, que, no âmbito da computação, significa o ato de fazer um mesmo programa/sistema/SO funcionar em diferentes ambientes. Por exemplo, fazer um software que antes só funcionava no Linux passar a funcionar em um outro SO (sistema operacional) pode ser considerado um porte. Uma palavra alternativa que poderia ser usada é ``suporte'', entretanto acredito que esta palavra não expresse apropriadamente o que foi feito, já que essa palavra normalmente é associada com um serviço pago de assistência técnica, e o próprio fraseamento do que foi feito se tornaria de mais difícil compreensão e prolixo com esta palavra.} do sistema operacional EPOS para a plataforma Zedboard, bem como comentar sobre as decisões tomadas e problemas enfrentados.

\section{Motivação}
EPOS é um sistema operacional que visa reduzir o custo de produção das aplicações embarcadas provendo um ambiente onde a aplicação do desenvolvedor possa rodar sem que ele tenha que lidar com configurações de baixo nível. Sendo assim é importante que o EPOS seja portado para um grande número de plataformas, do contrário sua utilidade prática seria prejudicada.

Zedboard é uma plataforma que usa o SoC Xilinx 7000, que por sua vez possui o processador \emph{dual core} ARM Cortex A9. Existe uma dupla motivação para se fazer o porte para esta plataforma: Primeiro que o EPOS ainda não foi portado para um processador RISC \emph{multicore}, e acredita-se que este porte poderá possibilitar linhas de pesquisa com o EPOS que antes só poderiam ser feitos num CISC, algo que não é o ideal considerando-se o nicho de aplicação do EPOS. O segundo fator está relacionado com o próprio Zedboard. Dentro do LISHA (Laboratório de Integração Software-Hardware) há uma linha de pesquisa sobre Smart Homes (ou Smart Buildings) em cujo contexto se pretende usar um Zedboard.

\section{Objetivos Gerais}

O objetivo deste trabalho é portar o sistema operacional EPOS e documentar este processo no presente documento. O porte consiste em adaptar os componentes que, dentro da arquitetura do EPOS, são chamados de mediadores de hardware. Cada mediador precisa ser refeito para cada plataforma.
Este trabalho terá certo direcionamento para as pessoas que no futuro precisem fazer um novo porte do EPOS, de modo que elas não precisem passar pelos mesmos problemas e possam, portanto, fazer isto mais eficientemente.

\section{Objetivos Específicos}
Dos objetivos específicos, pode-se organizá-los da seguinte forma:

\begin{enumerate}
    \item Estudo da arquitetura do EPOS, da arquitetura do ARM Cortex A9 e do Zedboard.
    \item Portes
    \begin{enumerate}
        \item Timers
        \item Controlador de Interrupção
        \item MMU
        \item CPU
		\item UART
        \item Inicialização elf e Multicore
    \end{enumerate}
    \item Validações
    \begin{enumerate}
        \item Escalonadores em single e multicore
        \item Threads
        \item Thread Periódica
        \item Alarmes
        \item Cronômetro
        \item Sincronização
    \end{enumerate}
    \item Documentação
\end{enumerate}
