\chapter{Conclusão}

%Falar dos algoritmos de escalonadores já implementados no EPOS, e que agora eles podem rodar no multicore implementado (preferencialmente teste isso, please)

%Fale de trabalhos futuros na conclusão! Frisar bastante os trabalhos que podem ser construídos em cima do porte, como o quadcoptero e a tese do Giovani.

Este trabalho visou descrever como foi feito o porte do sistema operacional EPOS para a Zedboard. Foram explicadas as decisões de projeto feitas, de modo que as próximas pessoas que venham a portar o EPOS para outra arquitetura, possam entender o raciocínio por trás do que foi implementado, para que o que foi aqui exposto possa ser adaptado a outros contextos.

Foi necessário explicar uma série de especificidades da arquitetura para expor como que é feito o interfaceamento do software com o hardware, e como um sistema operacional pode controlar e configurar componentes de hardware.

Este porte possibilitou novos cenários de aplicação do EPOS, espera-se portanto que isto contribua para estimular novas linhas de pesquisa com este sistema operacional na arquitetura trabalhada.
